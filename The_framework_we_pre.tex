The framework we present emphasizes the importance of accounting for global factors, that are reflected in the growth rate, when analyzing gene expression and proteomics data.
Specifically, we suggest that the default response of a protein (that is, the change in the observed expression of a protein, given that no specific regulation was applied to it) is to linearly increase with growth rate.
We note that the exact parameters of this dependency may depend on factors such as the degradation rate and the global rate of bio-synthesis mechanisms, as well as the specific affinity of the protein.
We point out that, as non-differentially regulated proteins maintain their relative abundances, one can overcome the lack of knowledge of these factors and use the scaling of most of the proteins in the proteome to infer these expected default dependency parameters \emph{ToDo:have figure demonstrating the connection between the response of a single protein and that of the global cluster or between two proteins that are members of the global cluster. this figure should also illustrate the reduction of parameters obtained by introducing the notion of affinities for expression}
