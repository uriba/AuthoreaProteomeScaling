When comparing the response of different proteins, a normalization is required to compensate for differences in their mean concentrations.
For example, consider two proteins, $A$ and $B$ measured under two conditions, $C_1$ and $C_2$.
Assume that the measured fractions out of the proteome of these two proteins under the two conditions were $0.001$ and $0.002$ for $A$ under $C_1$ and $C_2$ respectively, and $0.01$ and $0.02$ for $B$ under $C_1$ and $C_2$ respectively.
These two proteins therefore respond in the same way across the two conditions, namely, they double their fraction in the proteome in $C_2$ compared with $C_1$.
The normalization procedure scales the data so as to reveal this identity in response.
Thus, once the fraction of each protein out of the proteome is divided by the average fraction of that protein across conditions, we get that the normalized fractions are $\frac{2}{3}$ for both $A$ and $B$ at $C_1$ and $\frac{4}{3}$ for both $A$ and $B$ at $C_2$ showing their identical response.
