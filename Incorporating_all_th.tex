Incorporating all the condition-independent constants ($W_B$, $T_B$, $\ln(2)$) into one term, $C$, we get that the predicted fraction of protein $i$ out of the proteome under condition $c$ is:
\begin{equation}
  \label{eq:final-conc}
  p_i(c)=Cw_i(c)g(c)
\end{equation}
which implies that, for every two conditions between which gene $i$ maintains its affinity, ($w_i(c_1)=w_i(c_2)$), the fraction protein $i$ occupies out of the proteome scales like the growth rate change between these two conditions.
