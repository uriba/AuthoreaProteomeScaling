\subsection{Unannotated proteins in the Heinemann data set are less correlated with growth than annotated proteins}
Proteins with unknown function (present in Figure \ref{fig:growthcorr}, in the Heinemann data set, right panel) show less correlation with growth rate, as well as proteins with low levels of expression (data not shown).
\begin{comment}
\begin{figure}[h]
\centering
\includegraphics{CoordinatedRSquareComparison.pdf}
\caption{
  Proteins in the global cluster fit reasonably well to the global cluster itself
}
\label{fig:globalfit}
\end{figure}

\begin{figure}[h]
\centering
\includegraphics{GlobalClusterCorr.pdf}
\caption{
Proteins that have a high correlation (0.4-0.8) with growth rate mostly have even higher correlation to the sum of these proteins (both weighted sum and normalized sum are presented).
Weighted sum means the concentrations of all proteins in the group are summed.
Normalized sum means every protein is first normalized to have an average concentration of 1 across the different growth conditions, and then all proteins in the group are summed.
The higher correlation indicates that their response is coordinated (they scale by the same factor between conditions).
}
\label{globalcorr}
\end{figure}

\begin{figure}[h]
\centering
\includegraphics{GlobalClusterRSquare.pdf}
\caption{
Plotting the $r^2$ distribution shows that a large fraction of the variability of these proteins is captured by the global response.
}
\label{globalrsq}
\end{figure}

\end{comment}
