The resulting protein fraction, under a specific condition, is therefore its specific affinity under the condition, divided by the sum of all the affinities of all of the genes under that same condition.
Thus, if two proteins have the same affinity under some condition, they will occupy identical fractions out of the proteome under that condition.
If protein $A$ has twice the affinity of protein $B$ under a given condition, then the fraction $A$ occupies will be twice as large as the fraction occupied by $B$ under that condition, etc.
