As these genes form the bio-synthesis machinery, and according to the assumptions presented above, it follows that the doubling time under a given condition, $\tau(c)$ will be proportional to the ratio of total protein to bio-synthesis protein under that condition, with the proportionality constant being $T_B$:
\begin{equation}
  \label{eq:gr-ratio}
  \tau(c) = T_B\frac{P(c)}{\sum_{k\in G_B}P_k(c)}=T_B\frac{\sum_jw_j(c)}{W_B}
\end{equation}
Therefore, the model implies that for conditions that require the expression of larger amounts of non-bio-synthetic genes (i.e. higher values in the sum over $w_j$ that are not in $W_B$), the resulting doubling time will be longer, i.e., the growth rate will be lower.
