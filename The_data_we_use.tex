The data we use includes the concentrations of proteins under different growth conditions, and the growth rate for every condition.
Given a threshold on the Pearson correlation with growth rate, one can focus on the group of proteins with a correlation with growth rate that is higher than the threshold.
For this group of proteins, a linear regression response can be calculated.
We define the explained variability by the growth rate, given a threshold, as the difference between the total variability of the group of proteins with a correlation higher than the threshold, and the variability remaining, when assuming these proteins scale with the growth rate according to the calculated linear response.
Dividing the explained variability by the total variability of the entire data set quantifies what fraction of the total variability in the data set is explained by considering linear scaling with growth rate for all the proteins with a correlation with growth rate higher than the treshold.
The optimal threshold is then defined as the threshold maximizing this fraction.
Figure \ref{fig:threshold} shows the fraction of the variability explained as a function of the threshold used. 
