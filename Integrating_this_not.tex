Integrating this notion into the model means that, where the equations previously referred to the observed growth rate, $g$, as the indicator of protein synthesis rate, they should in fact refer to the observed growth rate plus the degradation rate, as that is the real rate of protein synthesis.
Therefore, if we denote by $\alpha$ the degradation rate (assuming for now equal degradation rates for all genes and under all conditions), Equation \ref{eq:final-conc} should be rewritten as:
\begin{equation}
  \label{eq:final-conc-deg}
  p_i(c)=Cw_i(c)(\alpha+g(c))
\end{equation}
This equation yields better agreement with the experimental results as presented in Figure \ref{fig:globalgrcorr}, depending on the exact value set for degradation.
Degradation can thus explain why the concentration of non-differentially regulated proteins does not drop to zero when the growth rate is zero.
The actual values obtained for the data analyzed are $\alpha=0.34$ for \cite{Valgepea2013} and $\alpha=0.44$ for \cite{Heinemann2014}, corresponding to protein half lives times of $T_{\text{deg}}=2$ hours and $T_{\text{deg}}=1.6$ hours respectively.
As these values correspond to relatively short half lives times, protein degradation is probably only a partial explanation for the differences between the predictions of the model and the observations obtained from the experimental data.
