Early on it was found that the expression of some genes is coordinated with growth rate, rather than with the specific environment.
Classical experiments in bacteria, by researchers from what became known as the Copenhagen school, have shown that ribosome concentration (inferred from the RNA to protein ratio in cells) increases in proportion to growth rate\cite{Schaechter1958}.
The search for mechanisms in \emph{E.coli} that underlie this observation yielded several candidates.
Specifically, coordination between ribosome production and growth rate was attributed both to the pools of purine nucleotides \cite{Gourse1996,Gaal1997}, and the tRNA pools through the stringent response \cite{Chatterji2001,Brauer2008a}.
For a more thorough review see \cite{Nomura1984}.
The logic behind this observed increase is that, given that translation rates and active ribosomes fraction remain relatively constant across conditions, a larger fraction of ribosomes out of the proteome is needed in order to achieve faster growth \cite{neidhardt1999a,dennis2004,Zaslaver2009}.
