In this study, we use the mass ratio of a specific protein to the mass of the entire proteome, per cell, as our basic measure for the bio-synthetic resources a specific protein consumes out of the bio-synthetic capacity of the cell.
We find this measure to be the best representation of the meaning of a fraction a protein occupies out of the proteome.
However, we note that if initiation rates are limiting, and not elongation rates, then using molecule counts ratios (the number of molecules of a specific protein divided by the total number of protein molecules in a cell) rather than mass ratios may be a better metric.
We compared these two metrics and, while they present some differences in the analysis, they do not qualitatively alter the observed results \emph{ToDo:see SI for comparison}.
