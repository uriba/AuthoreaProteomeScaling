Generalizing this notion, for every group of conditions, one could divide the proteins into those whose intrinsic affinity remains constant across all of the conditions, and to those whose intrinsic affinity changes (meaning their expression is actively regulated by the cell) between at least some of the conditions, as is shown in Figure \ref{fig:model}A.
An interesting consequence of the formulation in Equation \ref{eq:concentration-ratio} is that proteins whose intrinsic affinities remain constant across different growth conditions, also maintain their relative concentrations across these conditions with respect to each other.
Therefore, identifying a large group of proteins that maintain their relative concentrations across conditions (as was identified in section \ref{propchange}) may indicate that these proteins maintain their intrinsic affinities and that any changes in their absolute concentrations are in fact a passive outcome resulting from changes in the intrinsic affinities of other proteins.
