In many microorganisms, the proteome composition changes dramatically as a function of the growth environment.
Furthermore, many of these changes seem to be coordinated with the growth rate rather than the specific environment.
However, although cellular growth rates, gene expression levels and gene regulation have been at the center of biological research for decades, the quantitative interdependence between growth rate and proteome composition is not yet fully understood.

We analyzed the relationship between growth rate and proteome composition for the model microorganism \emph{E.coli} as reflected in two recently published proteomics data sets spanning various growth conditions.
We found that the cellular concentration of a large fraction of the proteins proportionally increases with the growth rate.
This fraction includes proteins that are involved in different cellular processes.
Notably, ribosomal proteins are only a small fraction of this group of proteins.
We present a simple model that demonstrates how such a widely coordinated, proportional, increase in the concentration of many proteins can be the result of passive redistribution of resources, due to active regulation of only a few other proteins.
Our model provides a potential explanation for why and how there is a global coordinated response of a large fraction of the proteome to the growth rate under different environmental conditions.
Its simplicity can also be useful by serving as a baseline null hypothesis in the search for active regulation.

We suggest that, although the concentrations of many proteins change with the growth rate, such changes could be part of a global effect, not requiring specific cellular control mechanisms.